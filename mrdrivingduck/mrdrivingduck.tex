%% start of file `template-zh.tex'.
%% Copyright 2006-2013 Xavier Danaux (xdanaux@gmail.com).
%
% This work may be distributed and/or modified under the
% conditions of the LaTeX Project Public License version 1.3c,
% available at http://www.latex-project.org/lppl/.


\documentclass[11pt,a4paper,sans]{moderncv}   % possible options include font size ('10pt', '11pt' and '12pt'), paper size ('a4paper', 'letterpaper', 'a5paper', 'legalpaper', 'executivepaper' and 'landscape') and font family ('sans' and 'roman')

% moderncv 主题
\moderncvstyle{banking}                        % 选项参数是 ‘casual’, ‘classic’, ‘oldstyle’ 和 ’banking’
\moderncvcolor{red}                          % 选项参数是 ‘blue’ (默认)、‘orange’、‘green’、‘red’、‘purple’ 和 ‘grey’
%\nopagenumbers{}                             % 消除注释以取消自动页码生成功能

% 字符编码
\usepackage[utf8]{inputenc}                   % 替换你正在使用的编码
\usepackage{CJKutf8}

% 调整页边距
\usepackage[scale=0.75]{geometry}
%\setlength{\hintscolumnwidth}{3cm}           % 如果你希望改变日期栏的宽度

% 个人信息
\name{张}{靖棠}
\title{个人简介}                     % 可选项、如不需要可删除本行
% \address{街道及门牌号}{邮编及城市}            % 可选项、如不需要可删除本行
\phone[mobile]{+86 15651700237}              % 可选项、如不需要可删除本行
% \phone[fixed]{+2~(345)~678~901}               % 可选项、如不需要可删除本行
% \phone[fax]{+3~(456)~789~012}                 % 可选项、如不需要可删除本行
\email{mrdrivingduck@gmail.com}                    % 可选项、如不需要可删除本行
\homepage{www.mrdrivingduck.cn}                  % 可选项、如不需要可删除本行
% \extrainfo{附加信息 (可选项)}                 % 可选项、如不需要可删除本行
% \photo[64pt][0.4pt]{picture}                  % ‘64pt’是图片必须压缩至的高度、‘0.4pt‘是图片边框的宽度 (如不需要可调节至0pt)、’picture‘ 是图片文件的名字;可选项、如不需要可删除本行
% \quote{引言(可选项)}                          % 可选项、如不需要可删除本行

\social[github][github.com/mrdrivingduck]{mrdrivingduck}
\social[linkedin][www.linkedin.com/in/jingtang-zhang-b38a76167/]{Jingtang Zhang}

% 显示索引号;仅用于在简历中使用了引言
%\makeatletter
%\renewcommand*{\bibliographyitemlabel}{\@biblabel{\arabic{enumiv}}}
%\makeatother

% 分类索引
%\usepackage{multibib}
%\newcites{book,misc}{{Books},{Others}}
%----------------------------------------------------------------------------------
%            内容
%----------------------------------------------------------------------------------
\begin{document}
\begin{CJK}{UTF8}{gbsn}                       % 详情参阅CJK文件包
\maketitle

\section{教育背景}
\cventry
{2019 年 -- (2022 年 4 月)}
{密码学与应用安全实验室 (CAS-Lab) 硕士学位在读}
{南京航空航天大学 计算机科学与技术学院 网络空间安全专业}
{江苏南京}{}{软件安全方向 程序缺陷自动修复 \newline{}}  % 第3到第6编码可留白
\cventry
{2015 年 -- 2019 年}
{本科学位}
{南京航空航天大学 计算机科学与技术学院 物联网工程专业}
{江苏南京}{}{GPA:4.2/5.0 专业排名:1/60}

% \section{毕业论文}
% \cvitem{题目}{\emph{题目}}
% \cvitem{导师}{导师}
% \cvitem{说明}{\small 论文简介}

% \section{工作背景}
% \subsection{专业}
% \cventry{年 -- 年}{职位}{公司}{城市}{}{不超过1--2行的概况说明\newline{}%
% 工作内容:%
% \begin{itemize}%
% \item 工作内容 1;
% \item 工作内容 2、 含二级内容:
%   \begin{itemize}%
%   \item 二级内容 (a);
%   \item 二级内容 (b)、含三级内容 (不建议使用);
%     \begin{itemize}
%     \item 三级内容 i;
%     \item 三级内容 ii;
%     \item 三级内容 iii;
%     \end{itemize}
%   \item 二级内容 (c);
%   \end{itemize}
% \item 工作内容 3。
% \end{itemize}}
% \cventry{年 -- 年}{职位}{公司}{城市}{}{说明行1\newline{}说明行2}
% \subsection{其他}
% \cventry{年 -- 年}{职位}{公司}{城市}{}{说明}

\section{竞赛获奖}

\cventry
{全国总决赛三等奖}
{
  \href
    {https://app.gitbook.com/@mrdrivingduck/s/linux-elf-sv/}
    {Linux ELF 可执行文件数据完整性保护系统}
}
{第九届 "中国软件杯" 大学生软件设计大赛}
{2020 年}
{}
{%
\begin{itemize}%
  \item \textbf{
    \href
      {https://github.com/NUAA-WatchDog/linux-kernel-elf-sig-verify-module}
      {ELF 可执行文件数字签名验证内核模块}
  }
  \item \textbf{
    \href
      {https://github.com/NUAA-WatchDog/linux-elf-binary-signer}
      {ELF 可执行文件数字签名工具}
  }\newline{}
\end{itemize}}

\cventry
{全国总决赛二等奖}
{基于物联网的智能车载健康系统}
{第七届 "中国软件杯" 大学生软件设计大赛}
{2018 年}
{}
{%
\begin{itemize}%
  \item 带有传感器与蓝牙模块的嵌入式系统
  \item \textbf{
    \href
    {https://github.com/NUAA-WatchDog/CARe-android}
    {智能车载健康 Android 客户端}
  }
  \item \textbf{
    \href
    {https://github.com/NUAA-WatchDog/CARe-cloud-http-server}
    {后端 HTTP 服务器}
  } 与多进程 SVM 模型管理器 \newline{}
\end{itemize}}

\cventry
{满分}
{C/C++ 甲级难度}
{浙江大学 2018 年春季 计算机程序设计能力考试 (PAT)}
{2018 年}
{}{排名 1/1459 \newline{}}

\cventry
{全国总决赛第七名}
{MIPS 静态单发射五级流水线 CPU}
{第一届 "龙芯杯" 全国大学生计算机系统能力培养大赛}
{2017 年}
{}{%
\begin{itemize}%
  \item \textbf{
    \href
    {https://github.com/mrdrivingduck/mips32-CPU}
    {在 Xilinx FPGA 上实现支持 57 条 MIPS 指令集的 CPU 电路}
  } \newline{}
\end{itemize}}

\cventry
{江苏赛区三等奖}
{C/C++ 程序设计大学 A 组}
{第八届 "蓝桥杯" 程序设计竞赛}
{2017 年}{}{}


\section{实习经历}

\cventry
{2018 年 7 月 -- 2018 年 8 月}
{软件开发工程师 (实习)}
{浙江智臾科技有限公司}
{浙江杭州}
{}{主要工作:
\begin{itemize}%
  \item \textbf{
    \href{https://github.com/dolphindb/api-r}{开发 DolphinDB R 语言 SDK}
  }
  \item \textbf{
    \href{https://zhuanlan.zhihu.com/p/42287416}{DolphinDB 与 InfluxDB 的性能评估测试}
  } \newline{}
\end{itemize}}


\section{其它项目经历}

\cventry{}
{负责前期技术选型和调研,实现了第一阶段的原型系统}
{华为云数据库 Oracle 存储过程迁移工具}
{2020 年}{}{}

\cventry{}
{负责调研无线嗅探器及入侵检测方法,以及系统集成}
{无线局域网入侵检测系统 (本科毕业设计)}
{2019 年}{}{
  \begin{itemize}%
    \item Vert.x + MongoDB 后端
    \item Vue.js 前端
    \item 开源的 kismet 无线嗅探器
    \item \textbf{
      \href
        {https://github.com/mrdrivingduck/kismet-Jclient}
        {Kismet 的 Java 客户端}
    }
    \item \textbf{
      \href
        {https://github.com/mrdrivingduck/scapy-web-server}
        {Python / Scapy 有线局域网嗅探器}} \newline{}
  \end{itemize}
}

\cventry
{结题等级优秀,答辩获一等奖}
{学院科创基金课题}
{基于云平台的校园智能门禁系统}
{2016 年}
{负责 Android 客户端开发}{}

\section{语言技能}
\cvitemwithcomment{英语}{CET6 (2016/06):547}{}
\cvitemwithcomment{英语}{CET4 (2015/12):584}{}
% \cvitemwithcomment{语言 2}{水平}{评价}
% \cvitemwithcomment{语言 3}{水平}{评价}

\section{所获荣誉}
\cvdoubleitem{国家奖学金}{2017}{优秀毕业生}{2019}
\cvdoubleitem{校级优秀学生奖学金一等奖}{2016,2017}{校级三好学生}{2016,2017,2018,2019}
\cvdoubleitem{校级优秀学生奖学金二等奖}{2018}{校级优秀学生干部}{2017,2018}
\cvdoubleitem{校级优秀学生奖学金三等奖}{2019}{学业奖学金一等奖}{2017,2018}

\section{业余爱好}
\cvitem{健身}{\small 业余水平}
\cvitem{羽毛球}{\small 曾参加 2017 年江苏省大学生羽毛球锦标赛}
\cvitem{阅读各类软件系统的源代码}{
  \small
  \href
    {https://github.com/mrdrivingduck/linux-kernel-comments-notes}
    {Linux 内核早期版本}、
  \href
    {https://github.com/mrdrivingduck/jdk-source-code-analysis}
    {JDK}、
  \href
    {https://github.com/mrdrivingduck/netty-in-action-notes}
    {Netty}、
  \href
    {https://github.com/mrdrivingduck/uc-os-ii-code-notes}
    {μC/OS-II}、
  \href
    {https://github.com/mrdrivingduck/understanding-nginx-notes}
    {Nginx}
等}
\cvitem{开发维护技术博客}{\small
  \href
    {https://www.mrdrivingduck.cn/blog/\#/}
    {https://www.mrdrivingduck.cn/blog/\#/}
}


\section{自我评价}
\cvitem{学习能力强,热爱技术}{\small 能够查阅各类中英文技术文档或论文,对软件的底层实现感兴趣}
\cvitem{自我要求高,责任心强}{\small 尤其是对代码,尽力追求完美,有时代码中少一个空格也会让我难受}
\cvitem{做事细心求稳}{\small 喜欢按部就班做事,思考完再动手,做事爆发力不强但后劲足}


\section{求职意向}
\cvitem{工作内容}{\small 偏系统研发}
\cvitem{工作地点}{\small 杭州}

% \section{其他 1}
% \cvlistitem{项目 1}
% \cvlistitem{项目 2}
% \cvlistitem{项目 3}

% \renewcommand{\listitemsymbol}{-}             % 改变列表符号

% \section{其他 2}
% \cvlistdoubleitem{项目 1}{项目 4}
% \cvlistdoubleitem{项目 2}{项目 5\cite{book1}}
% \cvlistdoubleitem{项目 3}{}

% 来自BibTeX文件但不使用multibib包的出版物
%\renewcommand*{\bibliographyitemlabel}{\@biblabel{\arabic{enumiv}}}% BibTeX的数字标签
% \nocite{*}
% \bibliographystyle{plain}
% \bibliography{publications}                    % 'publications' 是BibTeX文件的文件名

% 来自BibTeX文件并使用multibib包的出版物
%\section{出版物}
%\nocitebook{book1,book2}
%\bibliographystylebook{plain}
%\bibliographybook{publications}               % 'publications' 是BibTeX文件的文件名
%\nocitemisc{misc1,misc2,misc3}
%\bibliographystylemisc{plain}
%\bibliographymisc{publications}               % 'publications' 是BibTeX文件的文件名

\clearpage\end{CJK}
\end{document}


%% 文件结尾 `template-zh.tex'.
